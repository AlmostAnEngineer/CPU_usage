\documentclass[10pt,a4paper]{article}
\usepackage[utf8]{inputenc}
\usepackage{polski}
\usepackage{amsmath}
\usepackage{amsfonts}
\usepackage{amssymb}
\usepackage{graphicx}
\usepackage[left=2cm,right=2cm,top=2cm,bottom=2cm]{geometry}
\lstset{language=C,caption={Descriptive Caption Text},label=DescriptiveLabel}
\author{Jacek Pasiecki}
\title{cpu usage tracker}
\begin{document}
\maketitle
\section{Wstęp}
Zadanie rekrutacyjne dla firmy TietoEVRY. Udało się zrealizować większość założeń projektowych. W tym dokumencie opisze sposób analizowania danych wytycznych, sposób realizacji oraz inne pomysły w jaki sposób mógłbym to zrealizować. Sam projekt okazał się wyzwaniem oraz świetnym polem do rozszerzenia swoich umiejętności w C, niezwykle pomocna okazała się lektura ,,Modern C", dzięki której mogłem zapomnieć o tym co mnie nauczono na wykładach i na nowo uczyć się programowania w C. Projekt sprawdzono na linux'ie Manjaro oraz Ubuntu. Napisany został za pośrednictwem WSL2 w VisualStudio Code. Niestety praca na wielu komputerach (w czasie pisania tego programu miałem niestety ogromną ilość pracy na studiach oraz sporo zleceń) spowodowała błąd w systemie git, który nadpisał historię commitów. Z tego powodu postanowiłem utworzyć ten dokument, który mam nadzieję okaże się lepszą formą opisania pracy nad tym projektem.
\section{Funkcja main}
Założenia realizacji nie ulegały zmianie


\end{document}